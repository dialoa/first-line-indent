% Options for packages loaded elsewhere
\PassOptionsToPackage{unicode}{hyperref}
\PassOptionsToPackage{hyphens}{url}
%
\documentclass[
]{article}
\usepackage{amsmath,amssymb}
\usepackage{iftex}
\ifPDFTeX
  \usepackage[T1]{fontenc}
  \usepackage[utf8]{inputenc}
  \usepackage{textcomp} % provide euro and other symbols
\else % if luatex or xetex
  \usepackage{unicode-math} % this also loads fontspec
  \defaultfontfeatures{Scale=MatchLowercase}
  \defaultfontfeatures[\rmfamily]{Ligatures=TeX,Scale=1}
\fi
\usepackage{lmodern}
\ifPDFTeX\else
  % xetex/luatex font selection
\fi
% Use upquote if available, for straight quotes in verbatim environments
\IfFileExists{upquote.sty}{\usepackage{upquote}}{}
\IfFileExists{microtype.sty}{% use microtype if available
  \usepackage[]{microtype}
  \UseMicrotypeSet[protrusion]{basicmath} % disable protrusion for tt fonts
}{}
\usepackage{xcolor}
\usepackage{color}
\usepackage{fancyvrb}
\newcommand{\VerbBar}{|}
\newcommand{\VERB}{\Verb[commandchars=\\\{\}]}
\DefineVerbatimEnvironment{Highlighting}{Verbatim}{commandchars=\\\{\}}
% Add ',fontsize=\small' for more characters per line
\newenvironment{Shaded}{}{}
\newcommand{\AlertTok}[1]{\textcolor[rgb]{1.00,0.00,0.00}{\textbf{#1}}}
\newcommand{\AnnotationTok}[1]{\textcolor[rgb]{0.38,0.63,0.69}{\textbf{\textit{#1}}}}
\newcommand{\AttributeTok}[1]{\textcolor[rgb]{0.49,0.56,0.16}{#1}}
\newcommand{\BaseNTok}[1]{\textcolor[rgb]{0.25,0.63,0.44}{#1}}
\newcommand{\BuiltInTok}[1]{\textcolor[rgb]{0.00,0.50,0.00}{#1}}
\newcommand{\CharTok}[1]{\textcolor[rgb]{0.25,0.44,0.63}{#1}}
\newcommand{\CommentTok}[1]{\textcolor[rgb]{0.38,0.63,0.69}{\textit{#1}}}
\newcommand{\CommentVarTok}[1]{\textcolor[rgb]{0.38,0.63,0.69}{\textbf{\textit{#1}}}}
\newcommand{\ConstantTok}[1]{\textcolor[rgb]{0.53,0.00,0.00}{#1}}
\newcommand{\ControlFlowTok}[1]{\textcolor[rgb]{0.00,0.44,0.13}{\textbf{#1}}}
\newcommand{\DataTypeTok}[1]{\textcolor[rgb]{0.56,0.13,0.00}{#1}}
\newcommand{\DecValTok}[1]{\textcolor[rgb]{0.25,0.63,0.44}{#1}}
\newcommand{\DocumentationTok}[1]{\textcolor[rgb]{0.73,0.13,0.13}{\textit{#1}}}
\newcommand{\ErrorTok}[1]{\textcolor[rgb]{1.00,0.00,0.00}{\textbf{#1}}}
\newcommand{\ExtensionTok}[1]{#1}
\newcommand{\FloatTok}[1]{\textcolor[rgb]{0.25,0.63,0.44}{#1}}
\newcommand{\FunctionTok}[1]{\textcolor[rgb]{0.02,0.16,0.49}{#1}}
\newcommand{\ImportTok}[1]{\textcolor[rgb]{0.00,0.50,0.00}{\textbf{#1}}}
\newcommand{\InformationTok}[1]{\textcolor[rgb]{0.38,0.63,0.69}{\textbf{\textit{#1}}}}
\newcommand{\KeywordTok}[1]{\textcolor[rgb]{0.00,0.44,0.13}{\textbf{#1}}}
\newcommand{\NormalTok}[1]{#1}
\newcommand{\OperatorTok}[1]{\textcolor[rgb]{0.40,0.40,0.40}{#1}}
\newcommand{\OtherTok}[1]{\textcolor[rgb]{0.00,0.44,0.13}{#1}}
\newcommand{\PreprocessorTok}[1]{\textcolor[rgb]{0.74,0.48,0.00}{#1}}
\newcommand{\RegionMarkerTok}[1]{#1}
\newcommand{\SpecialCharTok}[1]{\textcolor[rgb]{0.25,0.44,0.63}{#1}}
\newcommand{\SpecialStringTok}[1]{\textcolor[rgb]{0.73,0.40,0.53}{#1}}
\newcommand{\StringTok}[1]{\textcolor[rgb]{0.25,0.44,0.63}{#1}}
\newcommand{\VariableTok}[1]{\textcolor[rgb]{0.10,0.09,0.49}{#1}}
\newcommand{\VerbatimStringTok}[1]{\textcolor[rgb]{0.25,0.44,0.63}{#1}}
\newcommand{\WarningTok}[1]{\textcolor[rgb]{0.38,0.63,0.69}{\textbf{\textit{#1}}}}
\usepackage{longtable,booktabs,array}
\usepackage{calc} % for calculating minipage widths
% Correct order of tables after \paragraph or \subparagraph
\usepackage{etoolbox}
\makeatletter
\patchcmd\longtable{\par}{\if@noskipsec\mbox{}\fi\par}{}{}
\makeatother
% Allow footnotes in longtable head/foot
\IfFileExists{footnotehyper.sty}{\usepackage{footnotehyper}}{\usepackage{footnote}}
\makesavenoteenv{longtable}
\setlength{\emergencystretch}{3em} % prevent overfull lines
\providecommand{\tightlist}{%
  \setlength{\itemsep}{0pt}\setlength{\parskip}{0pt}}
\setcounter{secnumdepth}{-\maxdimen} % remove section numbering
  \makeatletter
  \renewenvironment{quote}
     {\list{}{\listparindent 1.5em%
              \itemindent \listparindent
              \rightmargin \leftmargin
              \parsep \z@ \@plus \p@}%
            \item\noindent\relax}
      {\endlist}
  \makeatother
\setlength{\parindent}{2em}
\ifLuaTeX
  \usepackage{selnolig}  % disable illegal ligatures
\fi
\IfFileExists{bookmark.sty}{\usepackage{bookmark}}{\usepackage{hyperref}}
\IfFileExists{xurl.sty}{\usepackage{xurl}}{} % add URL line breaks if available
\urlstyle{same}
\hypersetup{
  pdftitle={First line indent},
  pdfauthor={Julien Dutant},
  hidelinks,
  pdfcreator={LaTeX via pandoc}}

\title{First line indent}
\author{Julien Dutant}
\date{22 Dec 2022}

\begin{document}
\maketitle

\noindent This document illustrates first-line indent typesetting. In
English typography, paragraphs just below a section heading aren't
indented, because a heading is enough to separate them from what is
before. The same should apply to the first paragraph of a document with
a title---so this paragraph is not indented.

This paragraph is indented. But after this quote:

\begin{quote}
\noindent Lorem ipsum dolor sit amet, consectetur adipiscing elit.
\end{quote}

\noindent the paragraph continues, so there should not be a first-line
indent.

We want this quote to end a paragraph:

\begin{quote}
\noindent Lorem ipsum dolor sit amet, consectetur adipiscing elit.
\end{quote}

\noindent The text below therefore begins a new paragraph and should
have a first-line indent. We have to manually specify using
\texttt{\textbackslash{}indent}.

\hypertarget{basic-tests}{%
\section{Basic tests}\label{basic-tests}}

After a heading (in English typographic style) the paragraph does not
have a first-line indent.

\hypertarget{manually-specifying-indentation-on-certain-paragraphs}{%
\subsection{Manually specifying indentation on certain
paragraphs}\label{manually-specifying-indentation-on-certain-paragraphs}}

In the couple couple of paragraphs that follow the quotes below, we have
manually specified \texttt{\textbackslash{}noindent} and
\texttt{\textbackslash{}indent} respectively. This is to check that the
filter doesn't add its own commands to those.

\begin{quote}
\noindent Lorem ipsum dolor sit amet, consectetur adipiscing elit.
\end{quote}

\noindent Here we've explicitly required a first line indent.

\noindent Here we've explicitly required \emph{not} to have one.

\hypertarget{automatic-removal-of-first-line-indentation}{%
\subsection{Automatic removal of first line
indentation}\label{automatic-removal-of-first-line-indentation}}

We can also check that indent is removed after lists:

\begin{itemize}
\tightlist
\item
  A bullet
\item
  list
\end{itemize}

\noindent And after code blocks:

\begin{Shaded}
\begin{Highlighting}[]
\KeywordTok{local} \VariableTok{variable} \OperatorTok{=} \StringTok{"value"}
\end{Highlighting}
\end{Shaded}

\noindent Or horizontal rules.

\begin{center}\rule{0.5\linewidth}{0.5pt}\end{center}

\noindent We can check that default behaviour is overridden for elements
of custom classes. We preserve indentation after certain code blocks:

\begin{Shaded}
\begin{Highlighting}[]
\NormalTok{This code block should be followed by an indented paragraph,}
\end{Highlighting}
\end{Shaded}

And we remove it after certain Divs:

This paragraph's Div container should not be followed by indentation,

\noindent as specified in the options.

\hypertarget{further-tests}{%
\section{Further tests}\label{further-tests}}

In this document we added a few custom filter options.

\hypertarget{size}{%
\subsection{Size}\label{size}}

The size of first-line indents is 2em instead of the default 1.5em
(Pandoc) or 1em (Quarto).

\hypertarget{keep-or-remove-indentation-after-certain-types-of-elements}{%
\subsection{Keep or remove indentation after certain types of
elements}\label{keep-or-remove-indentation-after-certain-types-of-elements}}

We also added an option to automatically remove indent after tables:

\begin{longtable}[]{@{}rlcl@{}}
\caption{Demonstration of simple table syntax.}\tabularnewline
\toprule\noalign{}
Right & Left & Center & Default \\
\midrule\noalign{}
\endfirsthead
\toprule\noalign{}
Right & Left & Center & Default \\
\midrule\noalign{}
\endhead
\bottomrule\noalign{}
\endlastfoot
12 & 12 & 12 & 12 \\
123 & 123 & 123 & 123 \\
1 & 1 & 1 & 1 \\
\end{longtable}

\noindent So this paragraph's first line is not indented. We added the
option \emph{not} to remove ident after ordered lists and definition
lists:

\begin{description}
\tightlist
\item[Definition]
This is a definition block.
\end{description}

So this paragraph is indented.

\begin{enumerate}
\def\labelenumi{\arabic{enumi}.}
\tightlist
\item
  An ordered
\item
  list
\end{enumerate}

And this one is too.

\hypertarget{recursion-and-nesting}{%
\subsection{Recursion and nesting}\label{recursion-and-nesting}}

The paragraphs below are nested within a Div element---actually, two
nested Divs, in order to check that the filter is applied recursively
within Divs.

\hypertarget{div}{}
\leavevmode\vadjust pre{\hypertarget{subdiv}{}}%
The first paragraph within a Div is indented normally, but the list
below

\begin{itemize}
\tightlist
\item
  list item
\item
  list item
\end{itemize}

\noindent should not be followed by a indented paragraph.

The last paragraph within Divs should be indented normally.

The filter is also applied recursively within blockquotes. A
blockquote's first paragraph shouldn't be indented, but any subsequent
ones should. Within the block quotes, indents should be removed after
special blocks, as in the main text.

\begin{quote}
\noindent The first paragraph of this blockquote does not have a first
line indent.

The subsequent paragraph has one. It's followed:

\begin{itemize}
\tightlist
\item
  by a
\item
  list
\end{itemize}

\noindent after which there is no first line indentation.

This next paragraph is first line indented again..
\end{quote}

\hypertarget{list-content}{%
\subsection{List content}\label{list-content}}

Within lists, paragraphs should be separated by vertical whitespace.

\begin{itemize}
\item
  This list item contains multiple paragraphs.

  The second one should not be indented, but separated by vertical
  whitespace.
\end{itemize}

\end{document}
